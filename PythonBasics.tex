\documentclass[11pt]{beamer}

% list of packages used
\usepackage[latin1]{inputenc}
\usepackage{amsmath}
\usepackage{amsfonts}
\usepackage{amssymb}
\usepackage{graphicx}
\usepackage{listings}

% Author details
\author{Sanyam, Nikhil @EduTechLabs}
\title{Pyhon Basics}

% configs

%%set theme 
\usetheme{Luebeck}
\usecolortheme{default}

%%configure listings package for python
% Default fixed font does not support bold face
\DeclareFixedFont{\ttb}{T1}{txtt}{bx}{n}{8pt} % for bold
\DeclareFixedFont{\ttm}{T1}{txtt}{m}{n}{8pt}  % for normal
\DeclareFixedFont{\ttc}{T1}{txtt}{i}{n}{7pt}  % for comments

%custom colours
\usepackage{color}
\definecolor{deepblue}{rgb}{0,0,0.5}
\definecolor{deepred}{rgb}{0.6,0,0}
\definecolor{deepgreen}{rgb}{0,0.5,0}

%configuring listings package to highlight python syntax
\newcommand\pythonstyle{\lstset{
language=Python,
basicstyle=\ttm,
otherkeywords={self,plot,run},             % Add keywords here
keywordstyle=\ttb\color{deepblue},
emph={MyClass,__init__},          % Custom highlighting
emphstyle=\ttb\color{deepred},    % Custom highlighting style
stringstyle=\color{deepgreen},
commentstyle=\ttc\color{gray},
%frame=tb,                         % Any extra options here
showstringspaces=false            % 
}}

% Python environment
\lstnewenvironment{python}[1][]
{
\pythonstyle
\lstset{#1}
}
{}

% Python for external files
\newcommand\pythonexternal[2][]{{
\pythonstyle
\lstinputlisting[#1]{#2}}}

% Python for inline
\newcommand\pythoninline[1]{{\pythonstyle\lstinline!#1!}}

%%%%%%%%%%%% begin writing the actual document %%%%%%%%%%%%%%%%%
\begin{document}

%get the title page
\frame{\maketitle}
% welcome page 
\begin{frame}{Welcome to the course}
	\begin{center}
	 Welcome to \emph{\color{deepblue}EduTechLab's} basic Python Course.
	\end{center}
\end{frame}

%%%%%%%%%%%%%%% about python

% preparing the code
\defverbatim[colored]\pycode{%
\begin{python}
class MyClass(Yourclass):
    def __init__(self, my, yours):
        bla = '5 1 2 3 4'
        print bla
\end{python}
}

\section{Intro to Python}
\subsection{Invoking Python}
% preparing the slide
\begin{frame}[containsverbatim]{The Python Interpreter}
	\begin{flushleft}
	\begin{enumerate}
		\item{Invoking the interpreter:} \\
		\pythoninline{python}
		\item{We like IPython}\\
		\pythoninline{ipython}
		\item{here is some more python code} \\
		\pycode
		\item{running a python script} \\
		in IPython
		\begin{python}
		run HelloWorld.py
		\end{python}
		from shell
		\begin{python}
		python HelloWorld.py
		\end{python}		
	\end{enumerate}
	\end{flushleft}
\end{frame}


%
%\begin{frame}[containsverbatim]{Defining python classes}
%\begin{python}
%class MyClass(Yourclass):
%    def __init__(self, my, yours):
%        bla bla bla...
%\end{python}
%\end{frame}
\subsection{Comments in Python}
\begin{frame}[containsverbatim]{An Informal Introduction to Python}
\begin{python}
# this is the first comment
spam = 1  # and this is the second comment
          # ... and now a third!
text = "# This is not a comment because it's inside quotes."
\end{python}
\end{frame}

\subsection{Numbers in Python}
\begin{frame}[containsverbatim]{Using Python as a Calculator}
\begin{python}
>>> 2 + 2
4
>>> 50 - 5*6
20
>>> (50 - 5.0*6) / 4
5.0
>>> 8 / 5.0
1.6
\end{python}
\end{frame}


\begin{frame}[containsverbatim]{Using Python as a Calculator}
Python is strictly typed
\begin{python}
>>> 17 / 3  # int / int -> int
5
>>> 17 / 3.0  # int / float -> float
5.666666666666667
>>> 17 // 3.0  # explicit floor division discards the fractional part
5.0
>>> 17 % 3  # the % operator returns the remainder of the division
2
>>> 5 * 3 + 2  # result * divisor + remainder
17
\end{python}
\end{frame}


\begin{frame}[containsverbatim]{Using Python as a Calculator}
\begin{python}
>>> 5 ** 2  # 5 squared
25
>>> 2 ** 7  # 2 to the power of 7
128
\end{python}
\end{frame}

\subsection{Variables in Python} 
\begin{frame}[containsverbatim]{Using Python as a Calculator}
\begin{python}
>>> tax = 12.5 / 100
>>> price = 100.50
>>> price * tax
12.5625
>>> price + _
113.0625
>>> round(_, 2)
113.06
\end{python}
\end{frame}

\subsection{Strings in Python}
\begin{frame}[containsverbatim]{Using Python as a Calculator}
Python strings are immutable
\begin{python}
>>> 'spam eggs'  # single quotes
'spam eggs'
>>> 'doesn\'t'  # use \' to escape the single quote...
"doesn't"
>>> "doesn't"  # ...or use double quotes instead
"doesn't"
>>> '"Yes," he said.'
'"Yes," he said.'
>>> "\"Yes,\" he said."
'"Yes," he said.'
>>> '"Isn\'t," she said.'
'"Isn\'t," she said.
\end{python}
\end{frame}

\subsection{The print statement}
\begin{frame}[containsverbatim]{Using Python as a Calculator}
\begin{python}
>>> print 'C:\some\name'  # here \n means newline!
C:\some
ame
>>> print r'C:\some\name'  # note the r before the quote
C:\some\name
\end{python}
\end{frame}

\section{Python string manipulation}
\subsection{String Concatenation}
\begin{frame}[containsverbatim]{Using Python as a Calculator}
\begin{python}
>>> # 3 times 'un', followed by 'ium'
>>> 3 * 'un' + 'ium'
'unununium'
>>> 'Py' 'thon'
'Python'
\end{python}
\end{frame}

\subsection{String splitting}
\begin{frame}[containsverbatim]{Using Python as a Calculator}
\begin{python}
>>> word = 'Python'
>>> word[0]  # character in position 0
'P'
>>> word[5]  # character in position 5
'n'
>>> word[-1]  # last character
'n'
>>> word[-2]  # second-last character
'o'
>>> word[-6]
'P'
>>> word[0:2]  # characters from position 0 (included) to 2 (excluded)
'Py'
>>> word[2:5]  # characters from position 2 (included) to 5 (excluded)
'tho'
\end{python}
\end{frame}

\section{Python Lists}

\subsection{List definition, slicing}
\begin{frame}[containsverbatim]{Using Python as a Calculator}
\begin{python}
>>> squares = [1, 4, 9, 16, 25]
>>> squares
[1, 4, 9, 16, 25]
>>> squares[0]  # indexing returns the item
1
>>> squares[-1]
25
>>> squares[-3:]  # slicing returns a new list
[9, 16, 25]
>>> squares[:]
[1, 4, 9, 16, 25]
>>> squares + [36, 49, 64, 81, 100]
[1, 4, 9, 16, 25, 36, 49, 64, 81, 100]
\end{python}
\end{frame}

\subsection{List modification - append values}
\begin{frame}[containsverbatim]{Using Python as a Calculator}
Lists are mutable
\begin{python}
>>> cubes = [1, 8, 27, 65, 125]  # something's wrong here
>>> 4 ** 3  # the cube of 4 is 64, not 65!
64
>>> cubes[3] = 64  # replace the wrong value
>>> cubes
[1, 8, 27, 64, 125]
>>> cubes.append(216)  # add the cube of 6
>>> cubes.append(7 ** 3)  # and the cube of 7
>>> cubes
[1, 8, 27, 64, 125, 216, 343]
\end{python}
\end{frame}

\subsection{List modification - replacing, removing}
\begin{frame}[containsverbatim]{Using Python as a Calculator}
\begin{python}
>>> letters = ['a', 'b', 'c', 'd', 'e', 'f', 'g']
>>> letters
['a', 'b', 'c', 'd', 'e', 'f', 'g']
>>> # replace some values
>>> letters[2:5] = ['C', 'D', 'E']
>>> letters
['a', 'b', 'C', 'D', 'E', 'f', 'g']
>>> # now remove them
>>> letters[2:5] = []
>>> letters
['a', 'b', 'f', 'g']
>>> # clear the list by replacing all the elements with an empty list
>>> letters[:] = []
>>> letters
[]
\end{python}
\end{frame}

\begin{frame}[containsverbatim]{Using Python as a Calculator}
\begin{python}
>>> letters = ['a', 'b', 'c', 'd']
>>> len(letters)
4
>>> a = ['a', 'b', 'c']
>>> n = [1, 2, 3]
>>> x = [a, n]
>>> x
[['a', 'b', 'c'], [1, 2, 3]]
>>> x[0]
['a', 'b', 'c']
>>> x[0][1]
'b'
\end{python}
\end{frame}

\section{Python Control Flow Tools}
\subsection{Assignment, while loop}
\begin{frame}[containsverbatim]{First Steps Towards Programming}
\begin{python}
>>> # Fibonacci series:
... # the sum of two elements defines the next
... a, b = 0, 1
>>> while b < 10:
...    print b
...    a, b = b, a+b
...
1
1
2
3
5
8
\end{python}
\end{frame}


\begin{frame}[containsverbatim]{First Steps Towards Programming}
\begin{python}
>>> a, b = 0, 1
>>> while b < 1000:
...    print b,
...    a, b = b, a+b
...
1 1 2 3 5 8 13 21 34 55 89 144 233 377 610 987
\end{python}
\end{frame}

\subsection{if, else, elif}
\begin{frame}[containsverbatim]{More Control Flow Tools}
\begin{python}
>>> x = int(raw_input("Please enter an integer: "))
Please enter an integer: 42
>>> if x < 0:
...    x = 0
...    print 'Negative changed to zero'
... elif x == 0:
...    print 'Zero'
... elif x == 1:
...    print 'Single'
... else:
...    print 'More'
...
More
\end{python}
\end{frame}

\subsection{For loop}
\begin{frame}[containsverbatim]{More Control Flow Tools}
Python for loops are a little different from Java and C
\begin{python}
>>> # Measure some strings:
... words = ['cat', 'window', 'defenestrate']
>>> for w in words:
...    print w, len(w)
...
cat 3
window 6
defenestrate 12
\end{python}
\end{frame}

\subsection{The range() Function}
\begin{frame}[containsverbatim]{More Control Flow Tools}
\begin{python}
>>> range(10)
[0, 1, 2, 3, 4, 5, 6, 7, 8, 9]
>>> range(5, 10)
[5, 6, 7, 8, 9]
>>> range(0, 10, 3)
[0, 3, 6, 9]
>>> range(-10, -100, -30)
[-10, -40, -70]
\end{python}
\end{frame}


\begin{frame}[containsverbatim]{More Control Flow Tools}
\begin{python}
>>> a = ['Mary', 'had', 'a', 'little', 'lamb']
>>> for i in range(len(a)):
...    print i, a[i]
...
0 Mary
1 had
2 a
3 little
4 lamb
\end{python}
\end{frame}

\subsection{break and continue, and else on Loops}
\begin{frame}[containsverbatim]{More Control Flow Tools}
\begin{python}
>>> for n in range(2, 10):
...    for x in range(2, n):
...        if n % x == 0:
...            print n, 'equals', x, '*', n/x
...            break
...    else: # We don't like this :P
...        # loop fell through without finding a factor
...        print n, 'is a prime number'
...
2 is a prime number
3 is a prime number
4 equals 2 * 2
5 is a prime number
6 equals 2 * 3
7 is a prime number
8 equals 2 * 4
9 equals 3 * 3
\end{python}
\end{frame}

\subsection{Functions}
\begin{frame}[containsverbatim]{Defining Functions}
\begin{python}
>>> def fib(n):    # write Fibonacci series up to n
...    """Print a Fibonacci series up to n."""
...    a, b = 0, 1
...    while a < n:
...        print a,
...        a, b = b, a+b
...
>>> # Now call the function we just defined:
... fib(2000)
0 1 1 2 3 5 8 13 21 34 55 89 144 233 377 610 987 1597

>>> fib # to show that python functions are also objects
<function fib at 10042ed0>
>>> f = fib
>>> f(100)
0 1 1 2 3 5 8 13 21 34 55 89
\end{python}
\end{frame}

\section{Python data types}
\subsection{Lists}
\begin{frame}[containsverbatim]{Data Structures}
\begin{python}
>>> a = [66.25, 333, 333, 1, 1234.5]
>>> print a.count(333), a.count(66.25), a.count('x')
2 1 0
>>> a.insert(2, -1)
>>> a.append(333)
>>> a
[66.25, 333, -1, 333, 1, 1234.5, 333]
>>> a.index(333)
1
>>> a.remove(333)
>>> a
[66.25, -1, 333, 1, 1234.5, 333]
>>> a.reverse()
>>> a
[333, 1234.5, 1, 333, -1, 66.25]
\end{python}
\end{frame}


\begin{frame}[containsverbatim]{Data Structures}
\begin{python}
>>> a.sort()
>>> a
[-1, 1, 66.25, 333, 333, 1234.5]
>>> a.pop()
1234.5
>>> a
[-1, 1, 66.25, 333, 333]
\end{python}
\end{frame}

\subsection{Using Lists as Stacks}
\begin{frame}[containsverbatim]{Data Structures}
\begin{python}
>>> stack = [3, 4, 5]
>>> stack.append(6)
>>> stack.append(7)
>>> stack
[3, 4, 5, 6, 7]
>>> stack.pop()
7
>>> stack
[3, 4, 5, 6]
>>> stack.pop()
6
>>> stack.pop()
5
>>> stack
[3, 4]
\end{python}
\end{frame}

\subsection{Using Lists as Queues}
\begin{frame}[containsverbatim]{Data Structures}
\begin{python}
>>> from collections import deque
>>> queue = deque(["Eric", "John", "Michael"])
>>> queue.append("Terry")           # Terry arrives
>>> queue.append("Graham")          # Graham arrives
>>> queue.popleft()                 # The first to arrive now leaves
'Eric'
>>> queue.popleft()                 # The second to arrive now leaves
'John'
>>> queue                      # Remaining queue in order of arrival
deque(['Michael', 'Terry', 'Graham'])
\end{python}
\end{frame}

\subsection{del statement}
\begin{frame}[containsverbatim]{Data Structures}
\begin{python}
>>> a = [-1, 1, 66.25, 333, 333, 1234.5]
>>> del a[0]
>>> a
[1, 66.25, 333, 333, 1234.5]
>>> del a[2:4]
>>> a
[1, 66.25, 1234.5]
>>> del a[:]
>>> a
[]
>>> del a
\end{python}
\end{frame}

\subsection{Tuples and Sequences}
\begin{frame}[containsverbatim]{Data Structures}
\begin{python}
>>> t = 12345, 54321, 'hello!'
>>> t[0]
12345
>>> t
(12345, 54321, 'hello!')
>>> # Tuples may be nested:
... u = t, (1, 2, 3, 4, 5)
>>> u
((12345, 54321, 'hello!'), (1, 2, 3, 4, 5))

>>> #tuples can contain mutable objects:
... v = ([1, 2, 3], [3, 2, 1])
>>> v
([1, 2, 3], [3, 2, 1])
\end{python}
\end{frame}


\begin{frame}[containsverbatim]{Data Structures}
Sets
\begin{python}
>>> basket = ['apple', 'orange', 'apple', 'pear', 'orange', 'banana']
>>> fruit = set(basket)       # create a set without duplicates
>>> fruit
set(['orange', 'pear', 'apple', 'banana'])
>>> 'orange' in fruit                 # fast membership testing
True
>>> 'crabgrass' in fruit
False

\end{python}
\end{frame}


\begin{frame}[containsverbatim]{Data Structures}
Dictionaries
\begin{python}
>>> tel = {'jack': 4098, 'sape': 4139}
>>> tel['guido'] = 4127
>>> tel
{'sape': 4139, 'guido': 4127, 'jack': 4098}
>>> tel['jack']
4098
>>> del tel['sape']
>>> tel['irv'] = 4127
>>> tel
{'guido': 4127, 'irv': 4127, 'jack': 4098}
>>> tel.keys()
['guido', 'irv', 'jack']
>>> 'guido' in tel
True
\end{python}
\end{frame}


\begin{frame}[containsverbatim]{Looping Techniques}
\begin{python}
>>> for i, v in enumerate(['tic', 'tac', 'toe']):
...    print i, v
...
0 tic
1 tac
2 toe
\end{python}
\end{frame}

\begin{frame}[containsverbatim]{Looping Techniques}
\begin{python}
>>> for i in reversed(xrange(1,10,2)):
...    print i
...
9
7
5
3
1
\end{python}
\end{frame}

\begin{frame}[containsverbatim]{Looping Techniques}
\begin{python}
>>>
>>> basket = ['apple', 'orange', 'apple', 'pear', 'orange', 'banana']
>>> for f in sorted(set(basket)):
...    print f
...
apple
banana
orange
pear
\end{python}
\end{frame}

\begin{frame}[containsverbatim]{Looping Techniques}
\begin{python}
>>> knights = {'gallahad': 'the pure', 'robin': 'the brave'}
>>> for k, v in knights.iteritems():
...    print k, v
...
gallahad the pure
robin the brave
\end{python}
\end{frame}

\begin{frame}[containsverbatim]{Input and Output}
Fancier Output Formatting
\begin{python}
>>> s = 'Hello, world.'
>>> str(s)
'Hello, world.'
>>> repr(s)
"'Hello, world.'"
>>> str(1.0/7.0)
'0.142857142857'
>>> repr(1.0/7.0)
'0.14285714285714285'
>>> x = 10 * 3.25
>>> y = 200 * 200
>>> s = 'The value of x is ' + repr(x) + ', and y is ' + repr(y) + '...'
>>> print s
The value of x is 32.5, and y is 40000...

\end{python}
\end{frame}

\begin{frame}[containsverbatim]{Reading and Writing Files}
\begin{python}
>>> f = open('workfile', 'w')
>>> print f
<open file 'workfile', mode 'w' at 80a0960>
\end{python}
\end{frame}

\begin{frame}[containsverbatim]{Reading and Writing Files}
Methods of File Objects
\begin{python}
>>> f.read()
'This is the entire file.\n'
>>> f.read()
''
>>> f.readline()
'This is the first line of the file.\n'
>>> f.readline()
'Second line of the file\n'
>>> f.readline()
''
>>> for line in f:
        print line,

This is the first line of the file.
Second line of the file
\end{python}
\end{frame}


\begin{frame}[containsverbatim]{Reading and Writing Files}
\begin{python}
>>> f.write('This is a test\n')
>>> value = ('the answer', 42)
>>> s = str(value)
>>> f.write(s)
>>> f = open('workfile', 'r+')
>>> f.write('0123456789abcdef')
>>> f.seek(5)     # Go to the 6th byte in the file
>>> f.read(1)
'5'
>>> f.seek(-3, 2) # Go to the 3rd byte before the end
>>> f.read(1)
'd'
\end{python}
\end{frame}


\begin{frame}[containsverbatim]{Reading and Writing Files}
\begin{python}
>>> f.close()
>>> f.read()
Traceback (most recent call last):
  File "<stdin>", line 1, in ?
ValueError: I/O operation on closed file
>>> with open('workfile', 'r') as f:
...    read_data = f.read()
>>> f.closed
True
\end{python}
\end{frame}

\begin{frame}[containsverbatim]{Saving structured data with json}
\begin{python}
>>> json.dumps([1, 'simple', 'list'])
'[1, "simple", "list"]'
json.dump(x, f)
x = json.load(f)
\end{python}
\end{frame}


\begin{frame}[containsverbatim]{Errors and Exceptions}
Syntax Errors
\begin{python}
>>> while True print 'Hello world'
  File "<stdin>", line 1, in ?
    while True print 'Hello world'
                   ^
SyntaxError: invalid syntax
\end{python}
\end{frame}


\begin{frame}[containsverbatim]{Errors and Exceptions}
Exceptions
\begin{python}
>>> 10 * (1/0)
Traceback (most recent call last):
  File "<stdin>", line 1, in ?
ZeroDivisionError: integer division or modulo by zero
>>> 4 + spam*3
Traceback (most recent call last):
  File "<stdin>", line 1, in ?
NameError: name 'spam' is not defined
>>> '2' + 2
Traceback (most recent call last):
  File "<stdin>", line 1, in ?
TypeError: cannot concatenate 'str' and 'int' objects
\end{python}
\end{frame}

\begin{frame}[containsverbatim]{Classes}
\begin{python}
Class Definition Syntax
class ClassName:
    <statement-1>
    .
    .
    .
    <statement-N>
\end{python}
\end{frame}

\begin{frame}[containsverbatim]{Classes}
Class Objects
\begin{python}
class MyClass:
    """A simple example class"""
    i = 12345
    def f(self):
        return 'hello world'

x = MyClass()

\end{python}
\end{frame}

\end{document}